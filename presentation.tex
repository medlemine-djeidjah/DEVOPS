\documentclass[aspectratio=169]{beamer}
\usepackage[french]{babel}
\usepackage[utf8]{inputenc}
\usepackage{hyperref}
\usepackage{graphicx}
\usepackage{xcolor}
\usepackage{tikz}
\usepackage{textcomp}

% Define custom colors
\definecolor{primaryColor}{RGB}{41, 128, 185}
\definecolor{secondaryColor}{RGB}{44, 62, 80}
\definecolor{accentColor}{RGB}{46, 204, 113}
\definecolor{bgColor}{RGB}{236, 240, 241}

% Theme configuration
\usetheme{Madrid}
\usecolortheme{whale}
\setbeamercolor{structure}{fg=primaryColor}
\setbeamercolor{title}{fg=white,bg=primaryColor}
\setbeamercolor{frametitle}{fg=primaryColor,bg=white}
\setbeamercolor{background canvas}{bg=bgColor}
\setbeamercolor{normal text}{fg=secondaryColor}
\setbeamercolor{item}{fg=accentColor}

% Custom footer
\makeatletter
\setbeamertemplate{footline}{
  \leavevmode%
  \hbox{%
    \begin{beamercolorbox}[wd=\paperwidth,ht=2.25ex,dp=1ex,center]{author in head/foot}%
      \usebeamerfont{author in head/foot}\insertshortdate\hspace*{2em} 
      Groupe 4: AHMED BOUHA, DJEIDIJAH, MOHAMMED SAAD\hspace*{2em} 
      \insertframenumber{} / \inserttotalframenumber
    \end{beamercolorbox}}%
  \vskip0pt%
}
\makeatother

% Title page information
\title{\textbf{Projet DevOps}}
\subtitle{Analyse et Implémentation d'une Pipeline CI/CD\\{\small Groupe 4}}
\author{
    Ahmed Abd Dayem AHMED BOUHA (23243) \\
    Mohammed Lemin MED DJEIDIJAH  (23244) \\
    Ahmed Ismail MOHAMMED SAAD (232422)
}
\date{\today}

\begin{document}

% Title frame
\begin{frame}
    \titlepage
\end{frame}

% Table of contents with description
\begin{frame}
    \frametitle{Vue d'Ensemble du Projet}
    \begin{columns}[t]
        \begin{column}{0.4\textwidth}
            \tableofcontents[hideallsubsections]
        \end{column}
        \begin{column}{0.6\textwidth}
            \textbf{Notre Mission:}\\
            Mettre en place une infrastructure DevOps moderne et robuste permettant un déploiement continu et sécurisé des applications.
            
            \vspace{0.5cm}
            \textbf{Objectifs Clés:}
            \begin{itemize}
                \item Automatisation complète
                \item Fiabilité maximale
                \item Maintenance simplifiée
            \end{itemize}
        \end{column}
    \end{columns}
\end{frame}

% Introduction with more context
\section{Introduction}
\begin{frame}
    \frametitle{Contexte du Projet}
    \begin{block}{Problématique}
        Comment orchestrer efficacement le déploiement de multiples applications interdépendantes tout en garantissant leur cohérence et leur stabilité?
    \end{block}
    \pause
    \begin{itemize}
        \item \textbf{Situation Actuelle:}
        \begin{itemize}
            \item Déploiements manuels risqués
            \item Temps d'intervention élevé
            \item Risques d'erreurs humaines
        \end{itemize}
        \pause
        \item \textbf{Notre Solution:}
        \begin{itemize}
            \item Pipeline CI/CD automatisé
            \item Gestion intelligente des versions
            \item Système de rollback automatique
        \end{itemize}
    \end{itemize}
\end{frame}

% Structure du Projet
\begin{frame}
    \frametitle{Structure du Projet}
    \begin{itemize}
        \item[-] DEVOPS/
        \begin{itemize}
            \item[+] apps/
            \begin{itemize}
                \item frontend/
                \item backend/
                \item database/
            \end{itemize}
            \item[+] ansible/
            \begin{itemize}
                \item playbooks/
                \begin{itemize}
                    \item setup\_docker.yml
                    \item deploy\_app.yml
                    \item rollback\_app.yml
                \end{itemize}
                \item inventory/
                \begin{itemize}
                    \item hosts.ini
                \end{itemize}
            \end{itemize}
            \item[-] Jenkinsfile
            \item[-] versions.yml
            \item[-] README.md
        \end{itemize}
    \end{itemize}
\end{frame}

% Architecture
\section{Architecture du Projet}
\begin{frame}
    \frametitle{Architecture du Projet}
    \begin{itemize}
        \item \textbf{Applications Déployées:}
        \begin{itemize}
            \item Frontend (Nginx)
            \item Backend (Python)
            \item Base de données (MySQL)
        \end{itemize}
        \item \textbf{Gestion des Versions:}
        \begin{itemize}
            \item Fichier versions.yml centralisé
            \item Versions indépendantes par service
        \end{itemize}
        \item \textbf{Infrastructure:}
        \begin{itemize}
            \item Jenkins pour l'orchestration
            \item Ansible pour le déploiement
            \item Conteneurisation Docker
        \end{itemize}
    \end{itemize}
\end{frame}

% Architecture Globale
\begin{frame}
    \frametitle{Architecture Globale}
    \begin{center}
    \begin{tikzpicture}[node distance=1.5cm]
        \node[draw, rectangle] (jenkins) {Jenkins Server};
        \node[draw, rectangle, below of=jenkins] (pipeline) {Pipeline CI/CD};
        \node[draw, rectangle, below of=pipeline] (versions) {versions.yml};
        \node[draw, rectangle, below of=versions] (ansible) {Ansible Playbooks};
        \node[draw, rectangle, below of=ansible, minimum width=4cm] (apps) {Frontend | Backend | DB};
        
        \draw[->] (jenkins) -- (pipeline);
        \draw[->] (pipeline) -- (versions);
        \draw[->] (versions) -- (ansible);
        \draw[->] (ansible) -- (apps);
    \end{tikzpicture}
    \end{center}
\end{frame}

% Flux de Déploiement
\begin{frame}
    \frametitle{Flux de Déploiement}
    \begin{enumerate}
        \item Déclenchement du Pipeline
        \item[\textrightarrow] Vérification versions.yml
        \item[\textrightarrow] Déploiement DB
        \item[\textrightarrow] Test DB $\rightarrow$ Si échec $\rightarrow$ Rollback
        \item[\textrightarrow] Déploiement Backend
        \item[\textrightarrow] Test Backend $\rightarrow$ Si échec $\rightarrow$ Rollback
        \item[\textrightarrow] Déploiement Frontend
        \item[\textrightarrow] Test Frontend $\rightarrow$ Si échec $\rightarrow$ Rollback
    \end{enumerate}
\end{frame}

% Détails Techniques
\begin{frame}
    \frametitle{Détails Techniques}
    \begin{columns}[t]
        \begin{column}{0.5\textwidth}
            \textbf{Infrastructure}
            \begin{itemize}
                \item Jenkins master
                \item Agents de déploiement
                \item Conteneurs Docker
                \item Réseau isolé
            \end{itemize}
        \end{column}
        \begin{column}{0.5\textwidth}
            \textbf{Sécurité}
            \begin{itemize}
                \item SSH avec clés
                \item HTTPS/TLS
                \item Secrets Jenkins
                \item Isolation réseau
            \end{itemize}
        \end{column}
    \end{columns}
\end{frame}

% Monitoring et Logs
\begin{frame}
    \frametitle{Monitoring et Logs}
    \begin{itemize}
        \item \textbf{Surveillance:}
        \begin{itemize}
            \item État des conteneurs
            \item Métriques système
            \item Temps de réponse
            \item Utilisation ressources
        \end{itemize}
        \item \textbf{Logs Centralisés:}
        \begin{itemize}
            \item Logs applicatifs
            \item Logs système
            \item Historique déploiements
            \item Traces de rollback
        \end{itemize}
    \end{itemize}
\end{frame}

% Points Forts du Projet
\begin{frame}
    \frametitle{Points Forts du Projet}
    \begin{itemize}
        \item \textbf{Automatisation Complète:}
        \begin{itemize}
            \item Déploiement sans intervention
            \item Tests automatisés
            \item Rollback intelligent
        \end{itemize}
        \item \textbf{Haute Disponibilité:}
        \begin{itemize}
            \item Zero-downtime deployment
            \item Basculement automatique
            \item Sauvegarde des données
        \end{itemize}
        \item \textbf{Maintenance Simplifiée:}
        \begin{itemize}
            \item Versions centralisées
            \item Documentation intégrée
            \item Processus standardisé
        \end{itemize}
    \end{itemize}
\end{frame}

% Jenkins Pipeline
\section{Pipeline Jenkins}
\begin{frame}
    \frametitle{Pipeline Jenkins}
    \begin{itemize}
        \item Intégration Continue (CI)
        \item Déploiement Continu (CD)
        \item Étapes automatisées
        \item Tests et qualité du code
    \end{itemize}
\end{frame}

% Jenkins Pipeline Details
\section{Pipeline Jenkins en Détail}
\begin{frame}
    \frametitle{Pipeline Jenkins en Détail}
    \begin{itemize}
        \item \textbf{Étapes du Pipeline:}
        \begin{itemize}
            \item Checkout du code
            \item Configuration Docker (conditionnelle)
            \item Déploiement des applications
            \item Gestion du rollback
        \end{itemize}
        \item \textbf{Paramètres Configurables:}
        \begin{itemize}
            \item Option de rollback
            \item Configuration Docker
        \end{itemize}
    \end{itemize}
\end{frame}

\begin{frame}
    \frametitle{Gestion des Versions et Rollback}
    \begin{itemize}
        \item \textbf{Fichier versions.yml:}
        \begin{itemize}
            \item nginx: latest
            \item mysql: latest
            \item backend: version 3.0
        \end{itemize}
        \item \textbf{Stratégie de Rollback:}
        \begin{itemize}
            \item Détection automatique des échecs
            \item Restauration cohérente des services
            \item Maintien de la compatibilité entre versions
        \end{itemize}
    \end{itemize}
\end{frame}

% Ansible Implementation
\section{Implémentation Ansible}
\begin{frame}
    \frametitle{Ansible: Automatisation et Déploiement}
    \begin{itemize}
        \item \textbf{Playbooks Principaux:}
        \begin{itemize}
            \item setup\_docker.yml: Installation et configuration
            \item deploy\_app.yml: Déploiement des services
            \item rollback\_app.yml: Gestion des rollbacks
        \end{itemize}
        \item \textbf{Caractéristiques:}
        \begin{itemize}
            \item Déploiement sans interruption
            \item Vérification de la santé des services
            \item Gestion des dépendances entre services
        \end{itemize}
    \end{itemize}
\end{frame}

% New Technical Details frame
\begin{frame}
    \frametitle{Spécifications Techniques}
    \begin{columns}[t]
        \begin{column}{0.5\textwidth}
            \begin{block}{Technologies Utilisées}
                \begin{itemize}
                    \item Jenkins (Pipeline)
                    \item Ansible (Configuration)
                    \item Docker (Conteneurisation)
                    \item Git (Versioning)
                \end{itemize}
            \end{block}
        \end{column}
        \begin{column}{0.5\textwidth}
            \begin{block}{Environnements}
                \begin{itemize}
                    \item Développement
                    \item Test (QA)
                    \item Production
                \end{itemize}
            \end{block}
        \end{column}
    \end{columns}
    \vspace{0.3cm}
    \begin{alertblock}{Points d'Attention}
        \begin{itemize}
            \item Sécurité renforcée
            \item Haute disponibilité
            \item Performance optimale
        \end{itemize}
    \end{alertblock}
\end{frame}

% Implementation Details
\begin{frame}
    \frametitle{Détails d'Implémentation}
    \begin{block}{Processus de Déploiement}
        \begin{enumerate}
            \item \textbf{Préparation:} Vérification des versions et dépendances
            \item \textbf{Base de Données:} Migration et sauvegarde
            \item \textbf{Backend:} Déploiement et tests
            \item \textbf{Frontend:} Mise à jour et validation
        \end{enumerate}
    \end{block}
    \pause
    \begin{block}{Mécanismes de Sécurité}
        \begin{itemize}
            \item Authentification SSH
            \item Chiffrement des données
            \item Isolation des environnements
        \end{itemize}
    \end{block}
\end{frame}

% Conclusion
\section{Conclusion}
\begin{frame}
    \frametitle{Conclusion}
    \begin{itemize}
        \item Résumé des réalisations
        \item Bénéfices du projet
        \item Perspectives futures
    \end{itemize}
\end{frame}

% Thank you slide
\begin{frame}
    \frametitle{Merci de votre attention}
    \begin{center}
        \Large{Questions?}
        \vspace{1cm}
        
        \textit{Projet réalisé par:}\\
        \medskip
        Ahmed Abd Dayem AHMEDBOUHA\\
        Mohammed Lemin MED djeidjah\\
        Ahmed Ismail Mohammed Saad
    \end{center}
\end{frame}

\end{document} 